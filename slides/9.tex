\begin{itemize}
	\item В термальном равновесии:
	\begin{equation*}
		C_A T_{\odot}^2 \approx
		9\cdot10^{-23} \text{s}	\left(\cfrac{   \avarage{\sigma_{a}v}    }{3\cdot10^{-26} \text{cm}^2\text{s}^{-1}} \right)
		\left(\cfrac{m_{\chi}}{\text{GeV}}\right)^{3/2}
	\end{equation*}
	\item В неупругом сценарии $a$ зависит от сечения рассения $\sigma_{\chi p}$, модели и времени и величина $C_A$ находится с помощью численного расчета линейного уравнения Больцмана.
	\item Учитывая изотропность задачи, уравнение эволюции на плотность ТМ $f(E,L)$ в пространстве $E$ --- $L$ следующее:
	\begin{equation*}
			\cderiv{f(E,L)}{t} = C(E,L)
			+ \int{dE'dL' S(E,L,E',L') f(E',L')}
	\end{equation*}
	где $C$ --- скорость захвата, а $S$ --- матрица столкновений.
\end{itemize}



