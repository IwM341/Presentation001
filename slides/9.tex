\begin{itemize}
	\item В упругом случае как правило $aT_{\odot}^2C >> 1$ и $A = C$. 
	\item В неупругом сценарии $a$ зависит от сечения рассения $\sigma_{\chi p}$, модели и времени.
	\item Величина $a$ находится с помощью численного расчета линейного уравнения Больцмана.
	\item Учитывая изотропность задачи, фазовое пространство --- плоскость $E$ --- $L$ и уравнение эволюции выглядит следующим образом:
\end{itemize}

\begin{equation*}
	\begin{split}
		\cderiv{f(E,L)}{t} = C(E,L)
		+ \int{dE'dL' S(E,L,E',L') f(E',L')}
	\end{split}
\end{equation*}
