\begin{itemize}
	\item Сечение рассеяния может быть независимым от спина ядра ($SI$) и зависимым ($SD$). 
	\item В первом случае когерентное рассеяние на $A$ нуклонах в ядре приводит росту сечения на $A^4$
	\begin{equation*}
		\sigma_{\chi N}(\hat{O}_1) = \sigma_{\chi p}\cdot A^4 \left(\cfrac{m_{\chi}+m_{p}}{m_{\chi} + m_{N}}\right)^2 (q^2 \rightarrow 0)
	\end{equation*}
	\item В случае $SD$ сечение растет только как $A^2$, из-за чего ограничения на сечение рассеяния слабее.
\end{itemize}