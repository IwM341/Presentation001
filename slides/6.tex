\begin{itemize}
	\item Взаимодействие тёмной материи представляется в виде линейной комбинации нерелятивистких операторов, возникающие из релятивистких операторов. Например:
	\begin{eqnarray*}
		\bar{\chi}\gamma^{\mu}\chi \bar{n}\gamma_{\mu}n \rightarrow& \hat{O}_1 &= 1 \\
		\bar{\chi}\gamma^{\mu}\gamma^{5}\chi \bar{n}\gamma_{\mu}\gamma^{5}n \rightarrow& -4\hat{O}_4  &= -4 \vec{S}_{\chi}\cdot\vec{S}_{n}
	\end{eqnarray*}
	\item Для нахождения сечения рассеяния на ядре находят в оболочечной модели ядра матричные элементы 
	потенциала взаимодействия.
	\begin{equation*}
		iV = \bra{\chi k',N p'} \sum_i{\hat{V}(r_{\chi} - r_{i})} \ket{\chi k,N p}
	\end{equation*}
\end{itemize}