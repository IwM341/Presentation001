\begin{itemize}
	\item Линейное уравнение на количество частиц в $i$-ом промежутке имеет вид:
	\begin{equation*}
		\deriv{N_{i}}{t} = \cfrac{1}{T_{\chi p}} \left(N_{\odot} c_i +
		\sum_j{[s_{ij} N_{j} - s_{ji} N_{i} ]} - e_{i} N_i  \right)
	\end{equation*}
	$T_{\chi p}^{-1} = \sigma_{\chi p} \avarage{n_{nuc}} v_{esc}$.
	\item Скорость захвата в $i$-ом интервале $c_i$, вероятности перехода/испарения $s_{ij}$/$e_{i}$ определяются интегралом столкновений по всем ядрам, учитывая модель Солнца (AGS09met).
\end{itemize}