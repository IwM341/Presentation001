Неупругая тёмная материя может естественно возникать в различных теориях.
\begin{itemize}
	\item Простейший пример --- дираковский фермиона малой майорановской массой
	\begin{equation*}
		\mathcal{L} \subset \overline{\chi}( i\gamma^{\mu} \partial_{\mu} - m)\chi + 
		\frac{\delta}{4} \overline{\chi} \chi^C + \frac{\delta}{4} \overline{\chi^C} \chi %+\frac{1}{2} m_2 \bar{\psi} \gamma^5 \psi^C 
	\end{equation*}
	Массовыми состояниями являются 
	\begin{eqnarray*}
		\chi_1 = \cfrac{\chi - \chi^C}{\sqrt{2} i} 
		%+O(\frac{\delta}{m_{\chi}})
		,\chi_2 = \cfrac{\chi+ \chi^C}{\sqrt{2}}
		%  + O(\frac{\delta}{m_{\chi}})
	\end{eqnarray*}
	с массами $m_1 = m - \frac{\delta}{2}$ и  $m_2 = m + \frac{\delta}{2}$
\end{itemize}